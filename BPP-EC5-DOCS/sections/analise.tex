\section{Análise de Desempenho}

Nesta seção, discutiremos a complexidade de tempo de cada uma das estratégias implementadas 
e faremos uma análise teórica sobre o impacto de diferentes tamanhos de vetor no tempo de execução.

\subsection{Complexidade de Tempo (Big-O)}

A complexidade de tempo de cada uma das estratégias de busca é analisada com base no número de 
operações realizadas em função do tamanho do vetor de entrada, denotado como \( n \).

\subsubsection{LinearSearchStrategy}

A busca linear percorre o vetor elemento por elemento até encontrar o valor procurado ou chegar 
ao final do vetor. Assim, no pior caso, ela realizará \( n \) comparações, onde \( n \) é o tamanho do vetor.

\textbf{Complexidade:} 
\[
O(n)
\]

Isso significa que, à medida que o tamanho do vetor aumenta, o tempo de execução cresce linearmente.

\subsubsection{BinarySearchStrategy}

A busca binária, por sua vez, é muito mais eficiente do que a busca linear. Ela começa com o vetor 
ordenado e divide o vetor ao meio a cada iteração, descartando metade dos elementos a cada passo. 
O número de comparações necessárias é reduzido drasticamente, resultando em uma complexidade de tempo logarítmica.

\textbf{Complexidade:}
\[
O(\log n)
\]

Neste caso, o tempo de execução cresce muito mais devagar à medida que o tamanho do vetor aumenta, 
pois o número de comparações diminui exponencialmente.

\subsection{Medições de Tempo de Execução}

Para comparar o desempenho das duas estratégias, realizamos medições de tempo em diferentes tamanhos de vetor. 
A seguir, discutimos os resultados esperados e como os tempos de execução variam com o tamanho do vetor para cada uma das estratégias.

\subsubsection{Resultados Esperados}

Como o comportamento das duas estratégias segue as complexidades \( O(n) \) e \( O(\log n) \), 
podemos esperar que, para vetores menores, a diferença no tempo de execução não seja tão significativa. 
No entanto, conforme o vetor cresce, a diferença se torna mais notável.

\subsubsection{Metodologia aplicada}

Para realizar os testes, foram criados arquivos de benchmark da ferramenta $Deno$, 
que realiza uma função diversas vezes para pegar a média, e retorna as informações bem
organizadas sobre o benchmark. A seguir temos um exemplo de uma função de benchmark do Deno:


\begin{verbatim}
Deno.bench("Linear Search", { group: "50" }, () => {
  linearBenchSearch(CINQUENTA);
});
\end{verbatim}

Para esses testes, foram feitas funções (linearBenchSearch e BinaryBenchSearch) 
que criam listas aleatórias e ordenadas com o tamanho passado como argumento,
inicializam as classes e então realizam a busca de todos os elementos daquea lista
pela estratégia escolhida. Segue exemplo da função linear:


\begin{verbatim}
export function linearBenchSearch(length: number) {
  const inventory: number[] = generateSortedRandomList(length);
  const linearSearch = new LinearSearchStrategy<number>();

  const ctx1 = new InventorySearchStrategy<number>(linearSearch, inventory);

  for (let i = 0; i < inventory.length; i++) {
    ctx1.executeSearch(inventory[i]);
  }
}
\end{verbatim}

\subsubsection{Resultado dos testes}

Abaixo estão os resultados de desempenho obtidos em benchmarks de busca linear e 
busca binária para diferentes tamanhos de vetores. As medições foram realizadas 
em uma máquina com CPU AMD Ryzen 5 5600 6-Core Processor e utilizando o runtime 
Deno 2.1.4 (x86\_64-unknown-linux-gnu).

\vspace{5mm}

\begin{tabular}{|l|l|}
\hline
\textbf{Benchmark} & \textbf{Tempo/Iteração (média)} \\
\hline
Binary Search & 6.2 µs \\
Linear Search & 6.8 µs \\
\hline
\end{tabular}

\vspace{1mm}

\textbf{Legenda:} Comparação em listas de tamanho 50.

\vspace{5mm}

\begin{tabular}{|l|l|}
\hline
\textbf{Benchmark} & \textbf{Tempo/Iteração (média)} \\
\hline
Binary Search & 183.0 µs \\
Linear Search & 739.8 µs \\
\hline
\end{tabular}

\textbf{Legenda:} Comparação em listas de tamanho 1000.

\vspace{5mm}

\begin{tabular}{|l|l|}
\hline
\textbf{Benchmark} & \textbf{Tempo/Iteração (média)} \\
\hline
Binary Search & 20.4 ms \\
Linear Search & 2.5 s \\
\hline
\end{tabular}

\textbf{Legenda:} Comparação em listas de tamanho 100.000.

\vspace{5mm}
