\section{Descrição do padrão Strategy}

O padrão de projeto Strategy é uma solução utilizada para encapsular diferentes
algoritmos ou comportamentos, permitindo que eles sejam selecionados dinamicamente 
em tempo de execução. Ele é composto por três elementos principais: a interface que 
define a estratégia, as implementações concretas das estratégias, e o contexto que 
utiliza uma dessas estratégias.

Neste projeto, a interface SearchStrategy representa a definição de uma estratégia 
de busca, exigindo que todas as implementações forneçam um método para buscar 
valores em um vetor de um tipo qualquer. As classes concretas, LinearSearchStrategy 
e BinarySearchStrategy, implementam essa interface, encapsulando os algoritmos de 
busca linear e binária, respectivamente.

A classe LinearSearchStrategy percorre sequencialmente o vetor em busca 
do valor desejado, sendo ideal para vetores pequenos ou não ordenados. 
Já a classe BinarySearchStrategy explora a ordenação dos elementos no vetor 
para realizar a busca de forma mais eficiente, dividindo o espaço de busca 
pela metade a cada iteração, mas exige que o vetor esteja ordenado.

Essa estrutura permite que o método de busca seja configurado de maneira 
flexível, possibilitando a troca entre diferentes estratégias sem modificar 
o código que utiliza a busca, alinhando-se ao princípio aberto/fechado 
(Open/Closed Principle) da programação orientada a objetos.