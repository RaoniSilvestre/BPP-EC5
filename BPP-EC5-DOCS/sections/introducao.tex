\section{Introdução}

Em sistemas de gerenciamento de inventário, é comum a necessidade
de localizar produtos específicos em coleções de dados, como vetores
que armazenam códigos de produtos. A eficiência dessas buscas desempenha
um papel fundamental, especialmente em sistemas com alta quantidade 
de registros ou que exigem respostas rápidas.

A busca em um vetor, por sua vez, pode ser realizada de diferentes maneiras,
sendo as mais comuns a busca linear e a busca binária. Enquanto a busca linear
verifica sequencialmente os elementos do vetor, a busca binária aproveita a
ordenação dos dados para reduzir o número de comparações pela metade a cada iteração.
Entretanto, cada abordagem apresenta vantagens e desvantagens que dependem do cenário 
de aplicação, como o tamanho do vetor e a ordenação dos dados.

Este relatório aborda a implementação de duas estratégias de busca em um vetor,
utilizando o padrão de projeto Strategy para permitir a troca dinâmica 
entre os métodos. Além disso, são realizados testes de desempenho para comparar 
as estratégias em diferentes tamanhos de vetores, discutindo os trade-offs entre 
eficiência e complexidade.