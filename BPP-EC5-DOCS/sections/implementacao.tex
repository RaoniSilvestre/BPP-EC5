\section{Implementação}

A implementação do padrão Strategy para busca é composta por três classes principais, uma interface e uma classe de contexto opcional. A seguir, é apresentada a estrutura básica e a lógica de cada parte:

\subsection{Classe SearchStrategy (Interface)}

A interface \texttt{SearchStrategy} define o método \texttt{search}, 
que será implementado pelas classes concretas. Esse método recebe um 
array de T e um valor de busca (\texttt{target}), 
retornando $true$ se o valor for encontrado ou $false$ caso o valor 
não seja encontrado.

\begin{verbatim}
export default interface SearchStrategy<T> {
    search(vec: T[], target: T): boolean
}
\end{verbatim}

\subsection{Classe LinearSearchStrategy}

A classe \texttt{LinearSearchStrategy} implementa a interface \texttt{SearchStrategy} 
com a busca linear. A busca linear percorre o array elemento por elemento, 
verificando se o valor procurado está presente.

\begin{verbatim}
export class LinearSearchStrategy<T> implements SearchStrategy<T> {
  constructor() {}

  search(vec: T[], target: T): boolean {
    vec.forEach((item) => {
      if (item === target) return true;
    });

    return false;
  }
}
\end{verbatim}

\subsection{Classe BinarySearchStrategy}

A classe \texttt{BinarySearchStrategy} também implementa a interface 
\texttt{SearchStrategy}, mas utiliza a busca binária, que é mais 
eficiente. A busca binária pressupõe que o array esteja ordenado. 
A estratégia consiste em dividir o array ao meio repetidamente e verificar 
em qual metade o valor pode estar, até encontrar o valor ou determinar 
que ele não está presente.

\begin{verbatim}
export class BinarySearchStrategy<T> implements SearchStrategy<T> {
    constructor() {}

    public search(vec: T[], target: T): boolean {
        let left = 0;
        let right = vec.length - 1;

        while (left <= right) {
            const mid = Math.floor((left + right) / 2);
            
            if (vec[mid] === target) {
                return true;
            }

            if (vec[mid] < target) {
                left = mid + 1;
            } else {
                right = mid - 1;
            }
        }

        return false;
    }
}
\end{verbatim}

\subsection{Classe de Contexto}

A classe de contexto \texttt{InventorySearchStrategy} facilita a troca 
de estratégias de busca em tempo de execução. Ela contém um campo 
\texttt{searchStrategy} que define a estratégia de busca atual e um método 
\texttt{setSearchStrategy} para alterar a estratégia. O método \texttt{executeSearch} 
é responsável por executar a busca utilizando a estratégia atual.

\begin{verbatim}
export class InventorySearchStrategy<T> {
    private searchStrategy: SearchStrategy<T>;
    private inventory: T[]

    constructor(searchStrategy: SearchStrategy<T>, inventory: T[]) {
        this.searchStrategy = searchStrategy
        this.inventory =inventory
    }

    setSearchStrategy(s: SearchStrategy<T>) { this.searchStrategy = s}

    executeSearch(target: T): boolean { 
        return this.searchStrategy.search(this.inventory, target)
    }
}
\end{verbatim}

\subsection{Exemplo de Uso}

No exemplo a seguir, um objeto \texttt{InventorySearchStrategy} é criado com a estratégia 
\texttt{LinearSearchStrategy}. O método \texttt{executeSearch} é chamado para 
buscar o valor \texttt{3} no array. Em seguida, a estratégia de busca é 
alterada para \texttt{BinarySearchStrategy}, e a busca é realizada novamente, 
mas em um array ordenado.

\begin{verbatim}
let context = new InventorySearchStrategy(new LinearSearchStrategy());
let arr = [5, 8, 1, 3, 9];
let target = 3;
console.log(context.executeSearch(arr, target)); // Resultado: 3

context.setStrategy(new BinarySearchStrategy());
arr = [1, 3, 5, 8, 9]; // O arr precisa estar ordenado para a busca binária
console.log(context.executeSearch(arr, target)); // Resultado: 3
\end{verbatim}

Essa implementação permite alternar entre diferentes estratégias de busca de 
maneira flexível, proporcionando uma solução escalável para diferentes cenários.
