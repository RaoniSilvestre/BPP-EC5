\section{Conclusão}

Neste estudo, comparamos o desempenho das estratégias de busca linear e busca binária para 
vetores de diferentes tamanhos. Os resultados mostraram que, para vetores pequenos (1000 elementos), 
a busca linear é ideal devido a sua simplicidade, visto que a binária não vai trazer muitas diferenças relevantes. 
No entanto, para vetores maiores, como os de 100,000 elementos, a busca binária se mostrou mais eficiente, 
devido a sua complexidade assintótica menor, ela performa muito bem para vetores ordenados.

Portanto, a escolha da estratégia de busca deve considerar o tamanho do vetor e o contexto 
específico da aplicação. Para vetores pequenos, a busca linear é vantajosa, enquanto para vetores 
grandes, a busca binária pode ser mais eficiente em termos de tempo de execução.
