\section{Discussão sobre Estrutura de Dados}

Com os resultados exibidos, nota-se que a medida que o tamanho da lista aumenta,
também aumenta muito o custo da busca linear, em comparação com a busca binária.

Entretando, vale ressaltar que a busca binária tem a limitação de só funcionar em
listar ordenadas. Desse modo, se não existe garantia da ordenação da lista, seria necessário
ordenar a lista para então buscar. 

Sabemos que o custo da ordenação nos melhores algoritmos é de $O(n \log n)$, então a complexidade
para ordenar e buscar (pior caso da busca binária), é inferior a busca linear. 

Com isso em mente, nota-se que caso seja sempre a mesma lista a ser buscada diversar vezes, sem 
necessidade de adicionar muitos elementos que desordenem a lista, vale a pena fazer uma única ordenação,
e então se beneficiar da busca binária nas buscas subsequentes. Caso não exista informação nenhuma sobre a 
lista a ser buscada, é melhor utilizar a busca linear.